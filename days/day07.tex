\chapter{Линковка, безопасность}

\section{Литература}
\textbf{elf} --- executable linked ...
Есть еще \textbf{dwarf}
Последняя книга будет интересна тем, кто хочет работать с безопасностью

\section{Превыполнение? (2 слайд)}
Процесс компиляции... (ассемблер, линковщик)

\section{Пример}
Код
Мы играем роль линковщика

\section{Ассемблер}
Правильные адреса расставляет линкер

\section{Объектный файл}
Текст --- функции
Данные --- переменные
Таблица символов --- карта символов

\section{Линковщик}
Занимается "склеиванием" объектных файлов
\shell{ld}
\shell{ld --verbose} --- показывает скрипт, который будет использоваться для линковки

\code{def.h}{C}
\code{main.c}{C}
\code{def.c}{C}
\code{Makefile}{C}
\shell{cat main.s}
\shell{cat def.s}
\shell{nm main.o}
\shell{nm def.o}
\textbf{T} --- text
\textbf{U} --- undefined (в другой единице трансляции)
\textbf{b} --- bss секция
\textbf{d} --- data
\shell{file main} --- Что за файл?
\shell{nm main}

Все склеиваем и не можем переиспользовать код
Тогда на помощь приходят статические библиотеки

\subsection{Статическая линковка}
\shell{ar rcs}
Единожды создаем статическую библиотеку и отдаем ее на линковку
Минусы: если есть уязвимость -> перекомпилировать все, находится в каждом файле
Плюсы: просто и понятно

\subsection{Динамическая линковка}
Экономия памяти
Не нужна пересборка
Первый минус --- откладываем линковку до времени \emph{runtime}
Второй минус --- ?
\shell{man ld.so} --- \emph{runtime} линкер
Две идеи --- 


\section{Linking vs Loading}
\textbf{Linking} --- связываем во что-то большее (\emph{compile time})
\textbf{Loading} --- загрузка в память (\emph{runtime})

\section{Динамическая линковка и PIC}
Картинка(got.png)
Картинка(offset.png)
\textbf{PLT/GOT}
Картинка(plt-before.png)
\textbf{PLT} находится в \emph{read-only} mode
Картинка(plt-after.png)
Зачем это нужно? Чтобы лениво подгружать
С динамическими библиотеками можно связываться явно и неявно
Явно --- \emph{dlclose, dlopen, dimopen, dlsym, dlvsym}
Неявно?:
\code{example1.c}{C}
\shell{cc example1.c}
В чем проблема?

\shell{cc example1.c -lpthread}
\shell{ldd ./a.out}
Надо указать, что нужно слинковаться
Тут еще \textbf{ASLR} работает

Еще пример
\code{example2.c}{C}
\shell{cc example2.c}
Не работает снова?
\shell{cc example2.c -ldl}

\textbf{ASLR} работает, но младшие байты остаются одними и те же
O_CLOEXEC --- флаг для безопасности
