Про сети

\section{Литература}
Книги хорошие, старые
CCNA --- примерно полугодовой курс. После сдачи сможете построить свою сеть

\section{Классификация (start)}
\begin{itemize} \item \textbf{Packet switched networks}\\ Эта концепция прижилась
    \item \textbf{Circut switched networks} --- коммутация каналлов (телефоны)\\
          Никто не может вмешаться в канал и повлиять на передачу\\
          Требует больше ресурсов.
\end{itemize}

\section{L2}
\subsection{Frame}
Картинка ethernet\_frame.jpg
\textbf{SFD} --- разграничитель
\textbf{Destination MAC} -- адрес получателя (идет первым, чтобы сетевая карта не делала лишнюю работу)
\textbf{Source MAC} --- адрес отправителя
\textbf{Type} --- какой протокол упакован внутри

\subsection{MAC адрес}
\begin{itemize}
    \item OUI\\
          Vendor assigned\\
          (все пространство MAC-адресов закреплено за какими-то производителями)
    \item \emph{Unicast} --- least signigicat bit of the first octet is 0\\
          Передача одному
    \item \emph{Multicast} --- least significant bit of the first octet is 1\\
          Передача многим
    \item \emph{Broadcast} --- ff:ff:ff:ff:ff:ff
          Передача всем \end{itemize} 
\subsection{Broadcast domain}
\textbf{VLAN} (trunk, access) --- виртуальная локальная сеть (используется поле \textbf{type})

\subsection{Devices}
\begin{itemize}
    \item \textbf{Hub}
        \begin{itemize}
            \item Дублирует электрический сигнал
            \item "Общая шина"
            \item Коллизии при одновременной отправке
            \item Алгоритм \emph{exponential backoff} --- каждый засыпаем на произвольное количество времени при коллизии
        \end{itemize}
    \item \textbf{Switch}
        \begin{itemize}
            \item Maps MAC to port (не биективно)
            \item Active learning (изучение)
            \item Flooding (лавинная рассылка)
            \item FIB --- база
        \end{itemize}
    \item \textbf{Bridge}
\end{itemize}

\subsection{Мысленный эксперимент}
\begin{enumerate}
    \item Чтобы кто-то отправил кому-то что-то, он должен знать его MAC-адрес (допустим, знает)
    \item Машина отправляет фрейм по MAC-адресу получателя
    \item Его получает коммутатор (L2 устройство), в нем адрес отправителя и адрес получателя
    \item Добавляет адрес отправителя в базу, если не сохранен порт получателя, начинается лавинная рассылка
    \item Получатель примет фрейм с теми же адресами отправителя и получателя
    \item Получатель отправляет ответ с реверснутыми адресами
    \item У коммутатора есть все данные и он конкретно отправляет фрейм отправителю первоначального фрейма. Также запоминает порт для получателя.
\end{enumerate}

\section{L3}
\textbf{L3} --- нужен чтобы передавать данные между сетями

\subsection{IP адрес}
\begin{itemize}
    \item \emph{Unicast}
    \item \emph{Multicast} --- 224.0.0/24
    \item \emph{Broadcast} --- 255.255.255.255
\end{itemize}

\subsection{CIDR}
Сначала были классы
\begin{itemize}
    \item A-сеть --- 1 старший байт
    \item B-сеть --- 2 старших байта
    \item C-сеть --- 3 старших байта
\end{itemize}
Потом классов не стало\\
\textbf{CIDR} (Classless InterDomain Routing) --- /8, /16, /24\\
Весь интернет: 0.0.0.0/32\\
Половины интернета: 0.0.0.0/1, 128.0.0.0/1\\
Конкретный адрес: *.*.*.*/32

\subsection{Кто заведует адресами?}
\begin{itemize}
    \item \textbf{IANA} --- main boss
    \item \textbf{RIR} --- организации на разных континентах
    \item \textbf{LIR} --- локальные организации
    \item \shell{whois}
    \item \textbf{AS} --- автономная система (инфраструктура, находящаяся по контролем)
\end{itemize}

\subsection{Mapping}
\textbf{ARP} (Address Resolution Protocol) --- кто имеет L3-адрес такой-то, скажи мне свой L2-адрес\\
\textbf{RARP} --- \todo{?}

\subsection{Routing}
К одному и тому же месту могут быть несколько маршрутов\\
\shell{netstat -rn} --- таблица маршрутизации\\
\shell{ip r}
\todo{
- Статическая
- Динамическая
* Distance vector
  - Bellman-Ford
* Link state
  - Dijkstra
}
\todo{
- IGP
  * RIP
  * OSPF
- EGP --- не работал
  * BGP --- работает
    - Full view
- stub --- только один маршрут наверх
- assymetric --- маршрут от отправителя к получателю необязательно совпадает с обратным маршрутом
- anycast --- одни и те же префиксы (в итоге получаем отказоустойчивость)
- ECMP (Equal Cost MultiParse) --- 
}
\todo{
 * RFC1918
}
\todo{
 * DHCP --- динамическое получение IP-адреса
}

\subsection{Devices}
\textbf{Router}

\subsection{Мысленный эксперимент}
\todo{Картинка с пояснениями}

\section{L4}
\subsection{Ports}
\begin{itemize}
    \item Priveleged
    \item Well known (/etc/services)
    \item Ephemeral
\end{itemize}

\subsection{Protocols}
\textbf{TCP} -- reliable, connection oriented
\begin{itemize}
    \item Header
    \item Handshaking
    \item Retransmits
    \item Congestion
    \item Timers
    \item HOL
    \item Active open / Passive open
    \item \todo{tcp-states.gif}
\end{itemize}
\textbf{UDP} --- unreliable, message oriented

\subsection{?}
\textbf{NAT} --- 

\todo{
  * Мысленный эксперимент
    - telnet vk.com 80
}

\section{L7}
\todo{}
  * DNS
    * Root servers
    * NS, A, PTR, MX
  * QUIC
  * HTTP
  * HTTP2
  * HTTP3
  * Мысленный эксперимент:
    - Открытие в браузере vk.com

\section{Misc}
\todo{}
  * Firewall
  * Application Firewall
  * DPI
  * NOC
  * tcpdump / wireshark
