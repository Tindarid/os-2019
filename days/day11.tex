Немного продолжения прошлой лекции

\section{L4}
\subsection{NAT}
\begin{itemize}
    \item Есть диапазоны сетей, которые не маршрутизируются (например, в двух компаниях одинаковые диапазоны)
    \item Позволили чуть дольше прожить с \textbf{Ipv4}
    \item \textbf{NAT} (Network Address Translation) --- технология (\textbf{L7}),  переводящая адреса, используя знания о уровнях (транспортный и т.д.) + переназначает порты
    \item Проблема с \textbf{FTP} (зашит порт)
    \item Поэтому в \textbf{NAT} добавлена функциональность deep-inspection (не очень deep)
\end{itemize}

\subsection{Разное}
\begin{itemize}
    \item \textbf{TTL} --- время жизни пакета, уменьшается на 1 при пересылке
    \item \textbf{Telnet} --- протокол, позволяющий удаленно подключаться к командной строке (данные видны всем)
    \item \textbf{SSH} --- криптография, все хорошо
    \item \shell{telnet}
    \item \shell{tcpdump} --- traffic sniffing
    \item \shell{wireshark} --- with GUI
    \item \shell{ping} --- с помощью протокола, который находится между 3 и 4 уровнями
    \item \shell{traceroute} --- путь пакета
    \item \shell{netstat -rn} --- таблица маршрутизации
    \item \shell{ip r}
\end{itemize}

\subsection{Мысленный эксперимент}
\shell{telnet vk.com 80}
\begin{itemize}
    \item Отправляем \textbf{SYN}
    \item \dots устанавливаем \textbf{TCP} соединение
    \item Различные \textbf{NAT}
\end{itemize}

\section{L7}
\begin{itemize}
    \item Люди не машины, числа очень сложно запоминать
    \item Сначала придумали /etc/hosts (непонятно как сопровождать)
\end{itemize}
\subsection{DNS}
\begin{itemize}
    \item Иерархическая база данных
    \item \shell{dig .}
    \item Корень знает ip-адреса серверов, в которых хранятся ip-адреса следующего уровня (root servers)
    \item \textbf{NS}, \textbf{A} --- Ipv4, \textbf{AAAA} --- Ipv6, \textbf{PTR}, \textbf{MX} --- почта
\end{itemize}
\subsection{Protocols}
\begin{itemize}
    \item \textbf{QUIC} --- новомодный транспортный протокол (базируется на UDP, находится в L7, т.к. создание нового протокола в kernelspace безумно дорого)
    \item \textbf{HTTP}
    \item \textbf{HTTP2}
    \item \textbf{HTTP3}
\end{itemize}

\section{Misc}
\begin{itemize}
    \item Firewall --- сервис, фильтрующий пакеты
    \item Application Firewall
    \item \textbf{DPI}
    \item \textbf{NOC} --- узкоспециализированные сисадмины
\end{itemize}
