Продолжение IPC (нет презентации, по книге)

\section{Pipes}
Гарантия атомарности упирается в константу (размер буфера)
Механизм \textbf{IPC} не дает гарантий границ сообщений
Если никто не пишет в \textbf{pipe}, то тот, кто читает из него --- блокируется (поток исполнения)

Есть неблокирующие файловые дескрипторы (вместо блокировки возвращают код ошибки)
\textbf{Globbing} --- подмножество регулярок (по маске получает что-то из файловой системы)

\subsection{Buffer}
*fix numbers of decriptors in lecture02 (0, 1, 2 ?)*
\emph{stdout} --- буферизированный, \emph{stderr} --- нет
Буферизация --- в данном случае, исключительно свойство библиотеки \textbf{libc} (системные вызовы не такие)

\code{stdout1.cpp}{C++}
Программа пишет, но ничего не выводится

\code{stdout2.cpp}{C++}
\textbf{std::endl} flushs output

\code{stdout3.cpp}{C++}
Не буферизированный

\subsection{Redirect}
\code{stdout4.cpp}{C++}
Выводы склеены
Редиректы позволяют рассортировать вывод (\shell{./a.out 2>err})
\textbf{Shell} обрабатывает перенаправления последовательно
\shell{./a.out 1>out 2>&1} (В конце перенаправление обоих в файл)
\shell{./a.out 2>&1 1>out} (Сначала и первый, и второй указывают на терминал, позже первый указывает на файл, в конце \emph{stderr} привязан к терминалу, а \emph{stdout} к файлу)
Еще бывают перенаправления ввода и вывода (\shell{wc -l < /etc/passwd})
Вся информация --- \shell{man sh}
\shell{man 3 open}
\shell{cat /etc/passwd | tee /tmp/out} --- одновременно выводим на экран и в файл

\section{System V}
Гарантирует то, что если вы записали сообщение, то его получат в том же размере
\textbf{Semaphores} --- like mutexes
\textbf{IPC Keys} --- just ints (\emph{ftok()})
\emph{msgget()}
Проблема семафоров --- слишком богатый интерфейс (можно сделать массив из семафоров, работать с ним паралелльно)
Симметричны в отношении к очередям ?
\code{systemv.cpp}{C++}

\section{Sockets}
\shell{man 2 socket}
\shell{man 7 ip}
Есть разные протоколы (транспортные, сетевые и т.д.)
Уровни (\todo{модели оси?}):
Userspace:
    L7 --- APPLICATION (\textbf{HTTP, MTproto, QUIC}, etc)
    L6, L5 --- SESSION, REPRESENTATION
             Чистая абстракция, способ поделить протокол на части, чтобы его проще было понимать
Kernelspace:
    L4 --- TRANSPORT (\textbf{TCP, UDP, SCTP}, etc)
    L3 --- NETWORK (\textbf{IPv4, IPv6, IPX})
    L2 --- \todo{?} (\textbf{802.3, 802.11})
    L1 --- PHYSICAL

\textbf{QUIC} --- новомодный транспортный протокол. Находится на последнем уровне для обратной совместимости.

\shell{man 2 connect}
