\chapter{IPC}

\todo{Ссылка на презентацию}

\section{Литература}
The Linux Programming Interface (практически все описано)

\section{Общее}
Есть много разных \textbf{IPC} --- Inter Process Communication
Примеры: \textbf{pipes in shell}, \textbf{sockets}, \textbf{System V shared memory},
\textbf{signals}, \textbf{mutexes}
\todo{Куча примеров из презентации}

\textbf{IPC} дает какой-то способ взаимодействия
\textbf{IPC} нужно выбирать с умом, зная требования к взаимодействию

Демоны (\textbf{daemons}) --- служебные процессы, которые долго живут 
и не перезагружаются (обслуживают что-то)

\section{Картинки}
\todo{Две картинки-схемы}

\section{Сигналы}
Сигналы характеризуются числом
\textbf{SIGSEGV} --- segmentation violation
\textbf{SIGBUS} --- генереируются в связи с проблемами маппинга виртуальной памяти на диск
\textbf{SIGINT} --- interrupt (Ctrl + C)
\textbf{SIGILL} --- illegal instruction
\textbf{SIGUSR1, SIGUSR2} --- отдаются на использование программисту
\textbf{SIGSTOP} --- процесс перестает шедулироваться (грубо говоря замораживается)
\textbf{SIGCONT} --- процесс начинает шедулироваться?
\shell{kill -SIGSTOP [number of proccess]}
Сигналы выставляются процессу

\code{signal1.cpp}{C++}
Что произошло?

У сигнала может быть три разных поведения: игнорирование, дефолтное, свой обработчик
\textbf{SIGKILL} и \textbf{SIGSTOP} --- нельзя ни перехватить, ни игнорировать
\textbf{SIGTERM} --- попросить процесс завершиться

Как послать сигнал самому себе?
\code{signal2.cpp}{C++}

Пишем свой обработчик сигнала
\code{signal3.cpp}{C++}
Нажимаем Ctrl + C, ловим сигнал

\code{signal4.cpp}{C++}
Генерация сигнала прерывает функцию \emph{sleep()}
Сигнал обрабатывается по границе выполняемой инструкции

\code{signal5.cpp}{C++}
Мы не имеем права звать из обработчика сигналов нереентерабельные функции (\emph{malloc()}, \emph{printf()}, \dots)

\code{signal6.cpp}{C++}
sig_atomic_t --- define для \todo{?}

\code{signal7.cpp}{C++}
Если сигнал возникнет в обработчике сигнала, то он обработается

Можно взять обработчик сигнала для \textbf{SIGIGN}, и поставить его также на обработку какого-нибудь другого

У интерфейса сигналов много проблем, поэтому появился \emph{advanced} интерфейс 
\shell{man sigaction}
Можно доставать из него информацию о проблеме (например, для \textbf{SIGSEGV} --- адрес памяти, которая защищена)

Гадание по \textbf{cr2} как в Матрице


\todo{Пример c ассемблером}


Если сигнал возник во время системного вызова, то он возвращается с кодом ошибки \textbf{EINTR}
\textbf{SA_RESTART} --- чтобы продолжить


\section{Pipes}
Примитив \textbf{IPC}
Данные на одном конце получаются ровно в том порядке, в котором они передаются с другого конца
\shell{man pipe}
\emph{pipe()} --- системный вызов для создания
\shell{man dup}
\emph{dup} --- создание копии файлового дескриптора
\textbf{pipe} == byte stream buffer in kernel

\section{FIFO}
Именованный \textbf{pipe}
\shell{mkfifo}
