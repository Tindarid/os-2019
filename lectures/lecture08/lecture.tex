\documentclass[../../lectures.tex]{subfiles}

\begin{document}

\chapter{API сокетов}

\section{Client and server}
\centerimage{server.png}{Simple client and server}{0.65}

\section{Sockets}
\begin{itemize}
    \item Абстракция, которая задает точку коммуникации
    \item \shell{man 2 socket}
    \item Можно выбрать протокол (\textbf{UDP}, \textbf{TCP}, \ldots)
    \item \emph{struct sockaddr} --- базовый класс в стиле Си для адресов протоколов
    \item Всегда выявляется активный клиент и пассивный сервер
\end{itemize}

\section{Server}
\begin{itemize}
    \item Создаем сокет
    \item Чтобы к нам могли подключаться --- мы должны заbindится на точку\\
          \shell{man 2 bind}
    \item Теперь мы можем принимать входящие вызовы (\shell{man 2 listen})
    \item \emph{backlog} --- пока меньше заданного значения, всем соединения будут 
          автоматически установлены (если больше, то будут игнорироваться) 
          --- механизм защиты
    \item \shell{man 2 accept} --- accept a connection on a socket
\end{itemize}
\centerimage{backlog.png}{Backlog}{0.65}

\section{Client}
\begin{itemize}
    \item \shell{man 2 connect} --- initiate a connection on a socket
\end{itemize}

\section{Sending information}
\begin{itemize}
    \item \shell{man 2 send}
    \item \shell{man 2 recv}
\end{itemize}

\section{Модели}
\begin{itemize}
    \item Все выше рассмотренные \emph{syscalls} --- блокирующие
    \item Это плохо: если есть 10 соединений и одно из них медленное
    \item \emph{fork}-модель --- есть master process, который для каждого соединения
          создает обслуживающий процесс
    \item Первые версии \textbf{Apache} так и работали
    \item Плюсы: скорость, безопасность, просто
    \item Минусы: дорого, не масштабируется
    \item Если делать \emph{concurrent}-модель, то будет слишком много потоков
\end{itemize}

\section{Cyris IMAP model}
\begin{enumerate}
    \item \emph{accept()}
    \item sockfd
    \item \emph{fork()}
    \item \emph{dup2()} stdin, stdout
    \item \emph{execve(бинарник)}
\end{enumerate}

\section{Multiplex}
\begin{itemize}
    \item \shell{man 2 select} --- не поддерживает больше 1024 дескрипторов\\
        \emph{Boost.AsIO} --- работает так
    \item 10k problem (just google it)
    \item \shell{man 2 poll} --- wait for some event on a fd
    \item Проблема \emph{poll()} в events
    \item \shell{man 2 epoll} --- разделение семантики \emph{poll()}
\end{itemize}

\section{Литература}
\begin{itemize}
    \item \shell{man 7 unix} --- Unix-sockets
    \item Stevens' Vol1
    \item \textcolor{blue}{
        \href{http://www.telecom.otago.ac.nz/tele402/lectures/lecture3.pdf}{Ссылка на презентацию}
    }
    \item \textcolor{blue}{
        \href{https://notes.shichao.io/unp/ch4}{See also}
    }
\end{itemize}

\section{Домашнее задание №6}
\textbf{Знакомство с мультиплексированием}
\begin{itemize}
    \item Необходимо попробовать клиент-серверное взаимодействие с использованием механизмов мультиплексирования.
    \item Помимо этого нужен \textbf{Makefile}, с помощью которого можно будет собрать клиент и сервер.
    \item Семейство протоколов для использования на выбор: AF\_UNIX, AF\_INET, AF\_INET6.
\end{itemize}

\textbf{Сервер должен:}
\begin{itemize}
    \item В качестве аргументов принимать адрес, на котором будет ожидать входящих соединений
    \item Стартовать, делать \emph{bind(2)} на заданный адрес и ожидать входящих соединений с использованием одного из механизмов мультиплексирования
    \item При получении соединения, добавлять дескриптор в механизм мультиплексирования и ожидать событий и на нем
    \item Выполнять на принятых соединениях серверную часть протокола
    \item По завершении обработки соединения удалять все события из механизмов мультиплексирования
\end{itemize}

\textbf{Клиент должен:}
\begin{itemize}
    \item Принимать параметром адрес, к которому стоит подключиться
    \item Используя механизмы мультиплексирования подключаться к серверу
    \item Используя механизмы мультиплексирования выполнять клиентскую часть протокола
    \item Завершаться
\end{itemize}

Для сильных духом предлагается реализовать код, который будет работать на двух разных ОС, используя на каждой специфичные механизмы мультиплексирования.
Сильность духа будет оцениваться в два балла.

\section{Домашнее задание №7}
\textbf{Знакомство с передачей дескрипторов и IPC}
\begin{itemize}
    \item Необходимо получить опыт работы с \textbf{IPC}. Нужно создать приложение клиента и сервера.
    \item Клиент и сервер обшаются через \textbf{UNIX} сокет
    \item Клиент подключается к серверу через \textbf{UNIX} сокет, получает от сервера файловый дескриптор, соответсвующий объекту какого-либо вида \textbf{IPC}
    \item Клиент и сервер выполняют какое-то взаимодействие используя \textbf{IPC}
\end{itemize}

\textbf{Сервер должен:}
\begin{itemize}
    \item Ожидать подключений на \textbf{UNIX} сокете
    \item Для новых соединений создавать новый вид \textbf{IPC}, объекты которого представимы в виде файловых дескрипторов
    \item Передавать через \textbf{UNIX} сокет клиенту файловый дескриптор \textbf{IPC} соответсвующий клиенту
    \item Ожидать выполнения какого-либо протокола поверх \textbf{IPC} с клиентом
\end{itemize}

\textbf{Клиент должен:}
\begin{itemize}
    \item Подключиться на UNIX сокет к серверу</li>
    \item Получать в виде файлового дескриптора клиентскую часть IPC</li>
    \item Взаимодействовать с сервером через IPC для выполнения какого-либо протокола
\end{itemize}

\textbf{Примерами IPC могут служить:}
\begin{itemize}
    \item Pipe
    \item Socket
    \item Файловые дескрипторы для разделямой памяти или сообщений(\textbf{POSIX})
    \item Файловые дескрипторы для анонимной памяти.
\end{itemize}

\textbf{Что может помочь при выполнении задания?}
\begin{itemize}
    \item \shell{man 7 unix}
\end{itemize}
\end{document}
