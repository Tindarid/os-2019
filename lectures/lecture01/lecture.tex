\documentclass[../../lectures.tex]{subfiles}

\begin{document}

\chapter{Введение}

\section{Преподаватель}
Банщиков Дмитрий Игоревич\\
\textbf{email:} me@ubique.spb.ru

\section{Операционные системы}
\begin{itemize}
    \item Операционная система --- это уровень абстракции между пользователем и машиной.
    Цель курса в том, чтобы объяснить, что происходит в системе от нажатия кнопки в
    браузере до получения результата.

    \item Курс будет посвящен Linux, потому что иначе говорить особо не о чем. Linux ---
    это операционная система общего назначения, для машин от самых маленьких почти
    без ресурсов до мощнейших серверов. Простой ответ почему Linux настолько
    популярен, а не Windows: в некоторых случаях он бесплатный.

    \item Почему полезно разрушить абстракцию черного ящика? Чтобы писать более
    оптимизированный и функциональный код. Иногда встречаются проблемы, которые не
    могут быть решены без знания внутренней работы ОС.
\end{itemize}

\section{Ядро и прочее}
\centerimage{kernel-scheme.png}{Схема Kernelspace и Userspace}{1}
\begin{itemize}
    \item Ядро Linux (\emph{kernel}) --- монолитное, это оправдано для ядра, но уязвимость одной части ядра
    ставит под угрозу все остальные части.

    \item Микроядерные ОС - альтернатива монолитным (мы не будем их изучать), но с ними
    сложно работать, потому что протоколы общения между частями требуют ресурсов.

    \item \emph{UNIX-like} системы - это системы, предоставляющие похожий на \emph{UNIX} интерфейс.
\end{itemize}
\end{document}
