\documentclass[../../lectures.tex]{subfiles}

\begin{document}
\chapter{Виртуальная память}

\section{Прерывания и исключения}
\begin{itemize}
    \item Процессор с памятью не могут быть жить в вакууме (без ОС)
    \item Простыми словами: иногда процессор не знает что ему делать 
          в конкретной ситуации, так как не знает контекста исполнения, 
          тогда он просит помощи у внешней среды (чаще всего ОС)
    \item \emph{Interrupt Deriving Architecture} --- есть таблица, каждой ячейке 
          которой соотвествует какая-либо исключительная ситуация(например, 
          поделить на нуль) и функция ее разрешающая.
    \item \textbf{IDTR} --- регистр, в котором хранится адрес 
          \emph{Interrupt Descriptor Table} (глобальный)

          В некоторых архитектурах находится по фиксированному адресу 
          (например, x86 --- защищенный режим)

          Проще говоря \textbf{callbacks}

    \item Другой подход --- \emph{Polling} (есть управляющий код, 
          который периодически опрашивает устройство на предмет того,
          что нужно обработать; процессор выставляет флаг --- 
          "нуждаюсь в обработке"). 
          
          Пример --- сетевая карта

    \item У каждого подхода свои плюсы и минусы

    \item \textbf{cr2} --- контрольный регистр, считывается функцией \emph{do\_page\_fault} 
          (которая занимается обработкой \textbf{page fault})
\end{itemize}

\section{Память}
Хочется чтобы каждый процесс был защищен от любого другого

Проблемы памяти:
\begin{enumerate}
    \item Памяти мало, она дорогая
    \item Памяти мало, программ много, как договориться?

          Закон Парето - 80\% обращений к 20\% памяти в среднем у пользовательской программы

          Может тогда выгружать неиспользуемую память на диск?

          Может еще переиспользовать память? (например, сегмент \textbf{text} у \textbf{Chrome})
    \item Памяти мало, программ много, как защититься?

          Неплохо было бы выложить в read-only какую-то память 
          (сегмент \textbf{text} --- \emph{const}-переменные)

          \code{stackoverflow.cpp}{C++}

          Способ защиты --- память, которая может записываться не может выполняться

          Канарейка на стеке --- проверяем значение переменной 'канарейка', которую добавили после адреса возврата

          Хотим чтобы память ядра была недоступна пользователям
\end{enumerate}
Как можно это все сделать?

\section{Подходы к организации памяти}
\todo{}
\subsection{Досегментная организация(№1)}
\subsection{Сегментная организация(№2)}
\subsection{Страничная организация(№3)}
\subsection{Страничная организация в x86(Реальность)}

\section{MMU}
\todo{}

\section{Переключение контекста}
\todo{}

\section{Page fault}
\todo{day06?}

\section{Литература}
\begin{enumerate}
    \item Understanding the Linux Kernel by Daniel P. Bovet \& Marco Cesati

          (Достаточно хорошо описана архитектура)
    \item Intel 64 and IA-32 Architectures Software Developer's Manual Volume 3

          (Руководство от Intel)
    \item x86 Instruction Set Architecture by Tom Shanley

          (Выжимка руководства от Intel)
    \item What every programmer should know about memory by Ulrich Drepper

          (Очень полезно)
    \item Безопасное программирование на C и C++. Роберт С. Сиакорд

          (Про уязвимости)
\end{enumerate}

\end{document}
