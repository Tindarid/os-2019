\documentclass[../../lectures.tex]{subfiles}

\begin{document}
\chapter{Виртуальная память}

\section{Прерывания и исключения}
\todo{}

\section{Память}
\todo{}

\section{Подходы к организации памяти}
\todo{}
\subsection{Досегментная организация}
\subsection{Сегментная организация}
\subsection{Страничная организация}
\subsection{Страничная организация в x86}

\section{MMU}
\todo{}

\section{Переключение контекста}
\todo{}

\section{Page fault}
\todo{day06?}

\section{Литература}
\begin{enumerate}
    \item Understanding the Linux Kernel by Daniel P. Bovet \& Marco Cesati

          (Достаточно хорошо описана архитектура)
    \item Intel 64 and IA-32 Architectures Software Developer's Manual Volume 3

          (Руководство от Intel)
    \item x86 Instruction Set Architecture by Tom Shanley

          (Выжимка руководства от Intel)
    \item What every programmer should know about memory by Ulrich Drepper

          (Очень полезно)
    \item Безопасное программирование на C и C++. Роберт С. Сиакорд

          (Про уязвимости)
\end{enumerate}

\end{document}
