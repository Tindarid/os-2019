\documentclass[../lectures.tex]{subfiles}

\begin{document}

\chapter{Файловые системы}

\section{Команды}
\begin{itemize}
    \item \shell{find} --- поиск
    \item \shell{ls} --- содержимое директории
\end{itemize}

\section{Носители}
\subsection{HDD}
\image{hard-disk.jpg}{Жесткий диск}
\begin{itemize}
    \item Обороты в минуту($O$) --- 5400, 7200, 10000, ...
    \item $~\frac{1}{2 * O}$ --- минимальное время доступа (случайное чтение)
    \item В мире Unix не существует дефрагментации (ОС должна сама заботиться)
    \item Время отказа (\textbf{MTBF} --- min time before failure) --- условное 
          количество циклов наработки до отказа
    \item На server --- сутки, desktop --- часы (разница в 3 раза примерно, 
          если одно и то же число циклов)
    \item Плюсы: стоимость, объем
    \item Минусы: время доступа, надежность
\end{itemize}

\subsection{Общее}
\begin{itemize}
    \item \textbf{EOPS} --- \todo{}
    \item \textbf{seek} --- рандомное чтение (512 байт)
    \item \textbf{SATA} и \textbf{NVME} --- протоколы для дисков
    \item \textbf{NVME} --- новомодная штука для \textbf{SSD}
    \item Минимум информации: сектор --- 512 байт -> 4096 байт
    \item Чтение одного байта равносильно чтению всего сектора с этим байтом
    \item Запись одного байта --- считать один сектор, заменить байт и записать один сектор
    \item Аналогия --- процессор-память --- \textbf{cacheline}
\end{itemize}

\section{Быстродействие}
\subsection{Интересные числа}
\begin{center}
Числа, которые должен знать каждый программист
\begin{tabular}{| l | l |}
    \hline
    Cycle                            & 1   ns \\ \hline
    Main memory reference            & 100 ns \\ \hline
    Read 4K randomly from SSD        & 150 us \\ \hline
    Read 1 MB sequentially from SSD  & 1   ms \\ \hline
    Disk seek                        & 10  ms \\ \hline
    Read 1 MB sequentially from disk & 20  ms \\ \hline
\end{tabular}
\end{center}
\subsection{Выводы для HDD}
\begin{itemize}
    \item Читать нужно последовательно
    \item Обращения к диску следует минимизировать
    \item Стоимость доступа сильно дороже передачи данных
\end{itemize}

\section{Structure packaging}
Сколько будет занимать памяти следующая структура?
\code{hole1.c}{C}
Ответ: 32 байта, так как $b$ и $d$ будут выравнены по MAX\_ALLIGNMENT

Очевидное решение проблемы:
\code{hole2.c}{C}
Данная структура будет занимать 24 байта на x86\_64.

\section{Алгоритмы элеватора}
\textcolor{blue}{\href{https://slideplayer.com/slide/5209336}{Ссылка на презентацию}}
\begin{enumerate}
    \item SLIDE 6
        
          Алгоритмы элеватора обрабатывают последовательности запросов к диску (переупорядочивают их)
    \item SLIDE 7

          \textbf{FCFS} (FIFO) --- самый простой и медленный
    \item SLIDE 8-9

          \textbf{SSTF} (Shortest Seek Time First)--- сортировка (очередной запрос определяется наименьшим временем seek)
    \item SLIDE 10 - \dots

          Различные способы упорядочивания
\end{enumerate}

\section{Файл}
\begin{itemize}
    \item Абстракция для данных
    \item Последовательность байтов
    \item Формат не определен
    \item \textbf{Unix} --- все есть файл (абстракция-интерфейс внутри ядра)
    \item Типы файлов 
          \begin{itemize}
            \item regular
            \item directory
            \item symlink
            \item socket, fifo
            \item character device, block device
          \end{itemize}
\end{itemize}

\section{Директория}
\begin{itemize}
    \item Содержит имена находящихся в нем файлов
    \item $.$ --- ссылка на текущую
    \item $..$ --- ссылка на родителя
    \item \shell{cd}, \shell{pwd}
    \item Формирование дерева: \shell{ls}
    \item \emph{filename vs pathname}: \shell{realpath}
    \item Права - "просто числа"
    \begin{itemize}
        \item \shell{view /etc/passwd}
        \item \shell{view /etc/group}
        \item \shell{id} - показывает идентификаторы того, кто ее вызывал
        \item \shell{execute} --- search
        \item \shell{read} --- directory listing
        \item \shell{write} --- changing directory
        \item Темные директории (переход в директорию внутри директории, для который ты не можешь посмотреть все файлы)
        \item \shell{chmod} --- меняет права доступа
    \end{itemize}
    \item \emph{sticky bit}
    \begin{itemize}
        \item Изменение поведения при создании нового файла
        \item /tmp
        \item Создаешь директорию со \emph{sticky bit} и все, кто создают файлы в этой директории имеют на них права
    \end{itemize}
\end{itemize}

\section{Иерархия}
\begin{itemize}
    \item $/$
        \begin{itemize}
            \item bin/
            \item dev/
            \item etc/
            \item sbin/
            \item home/
            \item var/
            \item usr/
                \begin{itemize}
                    \item bin/
                    \item sbin/
                \end{itemize}
            \item tmp
        \end{itemize}
\end{itemize}
\section{Монтирование}
\begin{itemize}
    \item Есть корень и есть узлы, в которые можно монтировать другие файловые системы (часть из них виртуальная)
    \item \shell{mount}
    \item Для $/$ обычно используется \textbf{ext4} (использует журналирование)
    \item Для /boot может использоваться \textbf{ext2} --- так как это более проверено временем (на Ubuntu)
    \item Файловая система для узла --- это не константа, ее можно менять
    \item \shell{df - h}, \shell{du -hs}
\end{itemize}
\section{Inode}
\todo{More from presentation}
\begin{itemize}
    \item Директория задает mapping имени файла в его inode
    \item \shell{ln}
    \item Hardlink --- существует в рамках одной файловой системы
    \item Softlink(symlink) --- бит l
    \item \shell{stat} --- информация о файле
    \item \emph{atime} --- время последнего доступа
    \item \emph{ctime} --- изменение мета-информации
    \item \emph{mtime}--- изменение содержимого файла
\end{itemize}
\end{document}
