\documentclass[../../lectures.tex]{subfiles}

\begin{document}

\chapter{IPC}

\section{Общее}
\begin{itemize}
    \item Есть много разных \textbf{IPC} --- Inter Process Communication
    \item Examples: \textbf{pipes in shell}, \textbf{sockets}, \textbf{System V shared memory},
          \textbf{signals}, \textbf{mutexes}
    \item \textbf{IPC} дает какой-то способ взаимодействия
    \item \textbf{IPC} нужно выбирать с умом, зная требования к взаимодействию
    \item Демоны (\textbf{daemons}) --- служебные процессы, которые долго живут 
          и не перезагружаются (обслуживают что-то)
\end{itemize}
\todo{Две картинки-схемы + классификация из презентации}

\newpage
\section{Сигналы}
\subsection{Общее}
\begin{itemize}
    \item Сигналы характеризуются числом
    \item \shell{kill -SIGSTOP [number of proccess]} --- послать сигнал процессу
    \item Сигналы выставляются процессу
\end{itemize}

Примеры:
\begin{itemize}
    \item \textbf{SIGSEGV} --- segmentation violation
    \item \textbf{SIGBUS} --- генерируется в связи с проблемами маппинга виртуальной памяти на диск \todo{Is it true?}
    \item \textbf{SIGINT} --- interrupt (Ctrl + C)
    \item \textbf{SIGILL} --- illegal instruction
    \item \textbf{SIGUSR1, SIGUSR2} --- отдаются на использование программисту
    \item \textbf{SIGSTOP} --- процесс перестает шедулироваться (грубо говоря замораживается)
    \item \textbf{SIGCONT} --- процесс начинает шедулироваться \todo{Is it true?}
    \item \textbf{SIGTERM} --- попросить процесс завершиться
\end{itemize}
\textbf{SIGKILL} и \textbf{SIGSTOP} --- нельзя ни перехватить, ни игнорировать

\newpage
\subsection{Пример №1}
Что произойдет?
\code{signal1.cpp}{C++}
У сигнала может быть три разных поведения: игнорирование, дефолтное, свой обработчик\\

\subsection{Пример №2}
Как послать сигнал самому себе?
\code{signal2.cpp}{C++}

\subsection{Пример №3}
Пишем свой обработчик сигнала
\code{signal3.cpp}{C++}
Нажимаем Ctrl + C, ловим сигнал

\subsection{Пример №4}
Сигнал обрабатывается по границе выполняемой инструкции
\code{signal4.cpp}{C++}
Генерация сигнала прерывает функцию \emph{sleep()}

\subsection{Пример №5}
\code{signal5.cpp}{C++}
Мы не имеем права звать из обработчика сигналов нереентерабельные функции (\emph{malloc()}, \emph{printf()}, \dots)

\subsection{Пример №6}
\code{signal6.cpp}{C++}
\textbf{sig\_atomic\_t} --- define для \todo{?}

\subsection{Пример №7}
\code{signal7.cpp}{C++}
Если сигнал возникнет в обработчике сигнала, то он обработается

\subsection{Пример №8}
\todo{Пример c ассемблером}

\subsection{Дополнительно}
\begin{itemize}
    \item Можно взять обработчик сигнала для \textbf{SIGIGN}, и поставить его также на обработку какого-нибудь другого

    \item У интерфейса сигналов много проблем, поэтому появился \emph{advanced} интерфейс\\
          \shell{man sigaction}\\
          Можно доставать из него информацию о проблеме (например, для \textbf{SIGSEGV} --- адрес памяти, которая защищена)

    \item \todo{Гадание по \textbf{CR2} как в Матрице}

    \item Если сигнал возник во время системного вызова, то он возвращается с кодом ошибки \textbf{EINTR}\\
          \textbf{SA\_RESTART} --- чтобы продолжить
\end{itemize}

\newpage
\section{Pipes}
\begin{itemize}
    \item Примитив \textbf{IPC}
    \item Данные на одном конце получаются ровно в том порядке, в котором они передаются с другого конца
    \item \emph{pipe()} --- системный вызов для создания\\
          \shell{man pipe}
    \item \emph{dup} --- создание копии файлового дескриптора\\
          \shell{man dup}
    \item \textbf{pipe} == byte stream buffer in kernel
    \item Гарантия атомарности упирается в константу (размер буфера)
    \item Механизм \textbf{IPC} не дает гарантий границ сообщений
    \item Если никто не пишет в \textbf{pipe}, то тот, кто читает из него --- блокируется (поток исполнения)

    \item Есть неблокирующие файловые дескрипторы (вместо блокировки возвращают код ошибки)
    \item \textbf{Globbing} --- подмножество регулярок (по маске получает что-то из файловой системы)
\end{itemize}

\subsection{Buffer}
\begin{itemize}
    \item \emph{stdout} --- буферизированный, \emph{stderr} --- нет
    \item Буферизация --- в данном случае, исключительно свойство библиотеки \textbf{libc} (системные вызовы не такие)
\end{itemize}

\subsubsection{Пример №1}
\code{pipe1.cpp}{C++}
Программа пишет, но ничего не выводится

\subsubsection{Пример №2}
\code{pipe2.cpp}{C++}
\textbf{std::endl} flushs output

\subsubsection{Пример №3}
\code{pipe3.cpp}{C++}
Не буферизированный

\subsection{Redirect}
\begin{itemize}
    \item Редиректы позволяют рассортировать вывод (\shell{./a.out 2>err})
    \item \textbf{Shell} обрабатывает перенаправления последовательно
        \begin{itemize}
                \item \shell{./a.out 1>out 2>\&1} (В конце перенаправление обоих в файл)
                \item \shell{./a.out 2>\&1 1>out} (Сначала и первый, и второй указывают на терминал, позже первый указывает на файл, в конце \emph{stderr} привязан к терминалу, а \emph{stdout} к файлу)
        \end{itemize}
    \item Еще бывают перенаправления ввода и вывода (\shell{wc -l < /etc/passwd})
    \item Вся информация --- \shell{man sh}
    \item \shell{man 3 open}
    \item \shell{cat /etc/passwd | tee /tmp/out} --- одновременно выводим на экран и в файл
\end{itemize}

\subsubsection{Пример}
\code{pipe4.cpp}{C++}
Выводы склеены

\section{FIFO}
Именованный \textbf{pipe} --- \shell{mkfifo}

\newpage
\section{SystemV}
\todo{More}
\begin{itemize}
    \item Гарантирует то, что если вы записали сообщение, то его получат в том же размере
    \item \textbf{Semaphores} --- like mutexes
    \item \textbf{IPC Keys} --- just ints (\emph{ftok()})
    \item \emph{msgget()}
    \item Проблема семафоров --- слишком богатый интерфейс (можно сделать массив из семафоров, работать с ним параллельно)
    \item Симметричны в отношении к очередям \todo{?}
\end{itemize}
\code{systemv.cpp}{C++}

\newpage
\section{Sockets}
\todo{More}
\begin{itemize}
    \item \shell{man 2 socket}
    \item \shell{man 7 ip}
    \item \shell{man 2 connect}
    \item Есть разные протоколы (транспортные, сетевые и т.д.)
\end{itemize}
Уровни (\todo{модели оси?}):

\textbf{Userspace:}
\begin{itemize}
    \item \textbf{L7} --- APPLICATION (\textbf{HTTP, MTproto, QUIC}, etc)
    \item \textbf{L6, L5} --- SESSION, REPRESENTATION

          Чистая абстракция, способ поделить протокол на части, чтобы его проще было понимать
\end{itemize}

\textbf{Kernelspace:}
\begin{itemize}
    \item \textbf{L4} --- TRANSPORT (\textbf{TCP, UDP, SCTP}, etc)
    \item \textbf{L3} --- NETWORK (\textbf{IPv4, IPv6, IPX})
    \item \textbf{L2} --- \todo{?} (\textbf{802.3, 802.11})
    \item \textbf{L1} --- PHYSICAL
\end{itemize}

\textbf{QUIC} --- новомодный транспортный протокол. Находится на последнем уровне для обратной совместимости.

\newpage
\section{Литература}
\begin{itemize}
    \item The Linux Programming Interface (практически все описано)
    \item \textcolor{blue}{
        \href{http://man7.org/conf/lca2013/IPC_Overview-LCA-2013-printable.pdf}{Ссылка на презентацию от автора книги}
    }
\end{itemize}

\end{document}
